% Generated by Sphinx.
\def\sphinxdocclass{report}
\documentclass[letterpaper,10pt,english]{sphinxmanual}
\usepackage[utf8]{inputenc}
\DeclareUnicodeCharacter{00A0}{\nobreakspace}
\usepackage{cmap}
\usepackage[T1]{fontenc}
\usepackage{babel}
\usepackage{times}
\usepackage[Bjarne]{fncychap}
\usepackage{longtable}
\usepackage{sphinx}
\usepackage{multirow}


\title{finance Documentation}
\date{November 25, 2013}
\release{}
\author{G. Barone, J. Bilbao de Mendizabal, A. Katre}
\newcommand{\sphinxlogo}{}
\renewcommand{\releasename}{Release}
\makeindex

\makeatletter
\def\PYG@reset{\let\PYG@it=\relax \let\PYG@bf=\relax%
    \let\PYG@ul=\relax \let\PYG@tc=\relax%
    \let\PYG@bc=\relax \let\PYG@ff=\relax}
\def\PYG@tok#1{\csname PYG@tok@#1\endcsname}
\def\PYG@toks#1+{\ifx\relax#1\empty\else%
    \PYG@tok{#1}\expandafter\PYG@toks\fi}
\def\PYG@do#1{\PYG@bc{\PYG@tc{\PYG@ul{%
    \PYG@it{\PYG@bf{\PYG@ff{#1}}}}}}}
\def\PYG#1#2{\PYG@reset\PYG@toks#1+\relax+\PYG@do{#2}}

\expandafter\def\csname PYG@tok@gd\endcsname{\def\PYG@tc##1{\textcolor[rgb]{0.63,0.00,0.00}{##1}}}
\expandafter\def\csname PYG@tok@gu\endcsname{\let\PYG@bf=\textbf\def\PYG@tc##1{\textcolor[rgb]{0.50,0.00,0.50}{##1}}}
\expandafter\def\csname PYG@tok@gt\endcsname{\def\PYG@tc##1{\textcolor[rgb]{0.00,0.27,0.87}{##1}}}
\expandafter\def\csname PYG@tok@gs\endcsname{\let\PYG@bf=\textbf}
\expandafter\def\csname PYG@tok@gr\endcsname{\def\PYG@tc##1{\textcolor[rgb]{1.00,0.00,0.00}{##1}}}
\expandafter\def\csname PYG@tok@cm\endcsname{\let\PYG@it=\textit\def\PYG@tc##1{\textcolor[rgb]{0.25,0.50,0.56}{##1}}}
\expandafter\def\csname PYG@tok@vg\endcsname{\def\PYG@tc##1{\textcolor[rgb]{0.73,0.38,0.84}{##1}}}
\expandafter\def\csname PYG@tok@m\endcsname{\def\PYG@tc##1{\textcolor[rgb]{0.13,0.50,0.31}{##1}}}
\expandafter\def\csname PYG@tok@mh\endcsname{\def\PYG@tc##1{\textcolor[rgb]{0.13,0.50,0.31}{##1}}}
\expandafter\def\csname PYG@tok@cs\endcsname{\def\PYG@tc##1{\textcolor[rgb]{0.25,0.50,0.56}{##1}}\def\PYG@bc##1{\setlength{\fboxsep}{0pt}\colorbox[rgb]{1.00,0.94,0.94}{\strut ##1}}}
\expandafter\def\csname PYG@tok@ge\endcsname{\let\PYG@it=\textit}
\expandafter\def\csname PYG@tok@vc\endcsname{\def\PYG@tc##1{\textcolor[rgb]{0.73,0.38,0.84}{##1}}}
\expandafter\def\csname PYG@tok@il\endcsname{\def\PYG@tc##1{\textcolor[rgb]{0.13,0.50,0.31}{##1}}}
\expandafter\def\csname PYG@tok@go\endcsname{\def\PYG@tc##1{\textcolor[rgb]{0.20,0.20,0.20}{##1}}}
\expandafter\def\csname PYG@tok@cp\endcsname{\def\PYG@tc##1{\textcolor[rgb]{0.00,0.44,0.13}{##1}}}
\expandafter\def\csname PYG@tok@gi\endcsname{\def\PYG@tc##1{\textcolor[rgb]{0.00,0.63,0.00}{##1}}}
\expandafter\def\csname PYG@tok@gh\endcsname{\let\PYG@bf=\textbf\def\PYG@tc##1{\textcolor[rgb]{0.00,0.00,0.50}{##1}}}
\expandafter\def\csname PYG@tok@ni\endcsname{\let\PYG@bf=\textbf\def\PYG@tc##1{\textcolor[rgb]{0.84,0.33,0.22}{##1}}}
\expandafter\def\csname PYG@tok@nl\endcsname{\let\PYG@bf=\textbf\def\PYG@tc##1{\textcolor[rgb]{0.00,0.13,0.44}{##1}}}
\expandafter\def\csname PYG@tok@nn\endcsname{\let\PYG@bf=\textbf\def\PYG@tc##1{\textcolor[rgb]{0.05,0.52,0.71}{##1}}}
\expandafter\def\csname PYG@tok@no\endcsname{\def\PYG@tc##1{\textcolor[rgb]{0.38,0.68,0.84}{##1}}}
\expandafter\def\csname PYG@tok@na\endcsname{\def\PYG@tc##1{\textcolor[rgb]{0.25,0.44,0.63}{##1}}}
\expandafter\def\csname PYG@tok@nb\endcsname{\def\PYG@tc##1{\textcolor[rgb]{0.00,0.44,0.13}{##1}}}
\expandafter\def\csname PYG@tok@nc\endcsname{\let\PYG@bf=\textbf\def\PYG@tc##1{\textcolor[rgb]{0.05,0.52,0.71}{##1}}}
\expandafter\def\csname PYG@tok@nd\endcsname{\let\PYG@bf=\textbf\def\PYG@tc##1{\textcolor[rgb]{0.33,0.33,0.33}{##1}}}
\expandafter\def\csname PYG@tok@ne\endcsname{\def\PYG@tc##1{\textcolor[rgb]{0.00,0.44,0.13}{##1}}}
\expandafter\def\csname PYG@tok@nf\endcsname{\def\PYG@tc##1{\textcolor[rgb]{0.02,0.16,0.49}{##1}}}
\expandafter\def\csname PYG@tok@si\endcsname{\let\PYG@it=\textit\def\PYG@tc##1{\textcolor[rgb]{0.44,0.63,0.82}{##1}}}
\expandafter\def\csname PYG@tok@s2\endcsname{\def\PYG@tc##1{\textcolor[rgb]{0.25,0.44,0.63}{##1}}}
\expandafter\def\csname PYG@tok@vi\endcsname{\def\PYG@tc##1{\textcolor[rgb]{0.73,0.38,0.84}{##1}}}
\expandafter\def\csname PYG@tok@nt\endcsname{\let\PYG@bf=\textbf\def\PYG@tc##1{\textcolor[rgb]{0.02,0.16,0.45}{##1}}}
\expandafter\def\csname PYG@tok@nv\endcsname{\def\PYG@tc##1{\textcolor[rgb]{0.73,0.38,0.84}{##1}}}
\expandafter\def\csname PYG@tok@s1\endcsname{\def\PYG@tc##1{\textcolor[rgb]{0.25,0.44,0.63}{##1}}}
\expandafter\def\csname PYG@tok@gp\endcsname{\let\PYG@bf=\textbf\def\PYG@tc##1{\textcolor[rgb]{0.78,0.36,0.04}{##1}}}
\expandafter\def\csname PYG@tok@sh\endcsname{\def\PYG@tc##1{\textcolor[rgb]{0.25,0.44,0.63}{##1}}}
\expandafter\def\csname PYG@tok@ow\endcsname{\let\PYG@bf=\textbf\def\PYG@tc##1{\textcolor[rgb]{0.00,0.44,0.13}{##1}}}
\expandafter\def\csname PYG@tok@sx\endcsname{\def\PYG@tc##1{\textcolor[rgb]{0.78,0.36,0.04}{##1}}}
\expandafter\def\csname PYG@tok@bp\endcsname{\def\PYG@tc##1{\textcolor[rgb]{0.00,0.44,0.13}{##1}}}
\expandafter\def\csname PYG@tok@c1\endcsname{\let\PYG@it=\textit\def\PYG@tc##1{\textcolor[rgb]{0.25,0.50,0.56}{##1}}}
\expandafter\def\csname PYG@tok@kc\endcsname{\let\PYG@bf=\textbf\def\PYG@tc##1{\textcolor[rgb]{0.00,0.44,0.13}{##1}}}
\expandafter\def\csname PYG@tok@c\endcsname{\let\PYG@it=\textit\def\PYG@tc##1{\textcolor[rgb]{0.25,0.50,0.56}{##1}}}
\expandafter\def\csname PYG@tok@mf\endcsname{\def\PYG@tc##1{\textcolor[rgb]{0.13,0.50,0.31}{##1}}}
\expandafter\def\csname PYG@tok@err\endcsname{\def\PYG@bc##1{\setlength{\fboxsep}{0pt}\fcolorbox[rgb]{1.00,0.00,0.00}{1,1,1}{\strut ##1}}}
\expandafter\def\csname PYG@tok@kd\endcsname{\let\PYG@bf=\textbf\def\PYG@tc##1{\textcolor[rgb]{0.00,0.44,0.13}{##1}}}
\expandafter\def\csname PYG@tok@ss\endcsname{\def\PYG@tc##1{\textcolor[rgb]{0.32,0.47,0.09}{##1}}}
\expandafter\def\csname PYG@tok@sr\endcsname{\def\PYG@tc##1{\textcolor[rgb]{0.14,0.33,0.53}{##1}}}
\expandafter\def\csname PYG@tok@mo\endcsname{\def\PYG@tc##1{\textcolor[rgb]{0.13,0.50,0.31}{##1}}}
\expandafter\def\csname PYG@tok@mi\endcsname{\def\PYG@tc##1{\textcolor[rgb]{0.13,0.50,0.31}{##1}}}
\expandafter\def\csname PYG@tok@kn\endcsname{\let\PYG@bf=\textbf\def\PYG@tc##1{\textcolor[rgb]{0.00,0.44,0.13}{##1}}}
\expandafter\def\csname PYG@tok@o\endcsname{\def\PYG@tc##1{\textcolor[rgb]{0.40,0.40,0.40}{##1}}}
\expandafter\def\csname PYG@tok@kr\endcsname{\let\PYG@bf=\textbf\def\PYG@tc##1{\textcolor[rgb]{0.00,0.44,0.13}{##1}}}
\expandafter\def\csname PYG@tok@s\endcsname{\def\PYG@tc##1{\textcolor[rgb]{0.25,0.44,0.63}{##1}}}
\expandafter\def\csname PYG@tok@kp\endcsname{\def\PYG@tc##1{\textcolor[rgb]{0.00,0.44,0.13}{##1}}}
\expandafter\def\csname PYG@tok@w\endcsname{\def\PYG@tc##1{\textcolor[rgb]{0.73,0.73,0.73}{##1}}}
\expandafter\def\csname PYG@tok@kt\endcsname{\def\PYG@tc##1{\textcolor[rgb]{0.56,0.13,0.00}{##1}}}
\expandafter\def\csname PYG@tok@sc\endcsname{\def\PYG@tc##1{\textcolor[rgb]{0.25,0.44,0.63}{##1}}}
\expandafter\def\csname PYG@tok@sb\endcsname{\def\PYG@tc##1{\textcolor[rgb]{0.25,0.44,0.63}{##1}}}
\expandafter\def\csname PYG@tok@k\endcsname{\let\PYG@bf=\textbf\def\PYG@tc##1{\textcolor[rgb]{0.00,0.44,0.13}{##1}}}
\expandafter\def\csname PYG@tok@se\endcsname{\let\PYG@bf=\textbf\def\PYG@tc##1{\textcolor[rgb]{0.25,0.44,0.63}{##1}}}
\expandafter\def\csname PYG@tok@sd\endcsname{\let\PYG@it=\textit\def\PYG@tc##1{\textcolor[rgb]{0.25,0.44,0.63}{##1}}}

\def\PYGZbs{\char`\\}
\def\PYGZus{\char`\_}
\def\PYGZob{\char`\{}
\def\PYGZcb{\char`\}}
\def\PYGZca{\char`\^}
\def\PYGZam{\char`\&}
\def\PYGZlt{\char`\<}
\def\PYGZgt{\char`\>}
\def\PYGZsh{\char`\#}
\def\PYGZpc{\char`\%}
\def\PYGZdl{\char`\$}
\def\PYGZhy{\char`\-}
\def\PYGZsq{\char`\'}
\def\PYGZdq{\char`\"}
\def\PYGZti{\char`\~}
% for compatibility with earlier versions
\def\PYGZat{@}
\def\PYGZlb{[}
\def\PYGZrb{]}
\makeatother

\begin{document}

\maketitle
\tableofcontents
\phantomsection\label{index::doc}


Contents:


\chapter{change\_stock module}
\label{change_stock::doc}\label{change_stock:welcome-to-finance-s-documentation}\label{change_stock:change-stock-module}

\chapter{conf module}
\label{conf:module-conf}\label{conf:conf-module}\label{conf::doc}\index{conf (module)}

\chapter{market module}
\label{market:module-market}\label{market::doc}\label{market:market-module}\index{market (module)}\index{market (class in market)}

\begin{fulllineitems}
\phantomsection\label{market:market.market}\pysiglinewithargsret{\strong{class }\code{market.}\bfcode{market}}{\emph{cap}}{}
Base for decribing stock evolution
\index{m\_cap (market.market attribute)}

\begin{fulllineitems}
\phantomsection\label{market:market.market.m_cap}\pysigline{\bfcode{m\_cap}\strong{ = 0}}
\end{fulllineitems}

\index{m\_time (market.market attribute)}

\begin{fulllineitems}
\phantomsection\label{market:market.market.m_time}\pysigline{\bfcode{m\_time}\strong{ = 0}}
\end{fulllineitems}


\end{fulllineitems}



\chapter{newfinance module}
\label{newfinance:module-newfinance}\label{newfinance:newfinance-module}\label{newfinance::doc}\index{newfinance (module)}
A collection of modules for collecting, analyzing and plotting
financial data.   User contributions welcome!
\index{candlestick() (in module newfinance)}

\begin{fulllineitems}
\phantomsection\label{newfinance:newfinance.candlestick}\pysiglinewithargsret{\code{newfinance.}\bfcode{candlestick}}{\emph{ax}, \emph{quotes}, \emph{width=0.2}, \emph{colorup='k'}, \emph{colordown='r'}, \emph{alpha=1.0}}{}
quotes is a sequence of (time, open, close, high, low, ...) sequences.
As long as the first 5 elements are these values,
the record can be as long as you want (eg it may store volume).

time must be in float days format - see date2num

Plot the time, open, close, high, low as a vertical line ranging
from low to high.  Use a rectangular bar to represent the
open-close span.  If close \textgreater{}= open, use colorup to color the bar,
otherwise use colordown

ax          : an Axes instance to plot to
width       : fraction of a day for the rectangle width
colorup     : the color of the rectangle where close \textgreater{}= open
colordown   : the color of the rectangle where close \textless{}  open
alpha       : the rectangle alpha level

return value is lines, patches where lines is a list of lines
added and patches is a list of the rectangle patches added

\end{fulllineitems}

\index{candlestick2() (in module newfinance)}

\begin{fulllineitems}
\phantomsection\label{newfinance:newfinance.candlestick2}\pysiglinewithargsret{\code{newfinance.}\bfcode{candlestick2}}{\emph{ax}, \emph{opens}, \emph{closes}, \emph{highs}, \emph{lows}, \emph{width=4}, \emph{colorup='k'}, \emph{colordown='r'}, \emph{alpha=0.75}}{}
Represent the open, close as a bar line and high low range as a
vertical line.

ax          : an Axes instance to plot to
width       : the bar width in points
colorup     : the color of the lines where close \textgreater{}= open
colordown   : the color of the lines where close \textless{}  open
alpha       : bar transparency

return value is lineCollection, barCollection

\end{fulllineitems}

\index{fetch\_historical\_yahoo() (in module newfinance)}

\begin{fulllineitems}
\phantomsection\label{newfinance:newfinance.fetch_historical_yahoo}\pysiglinewithargsret{\code{newfinance.}\bfcode{fetch\_historical\_yahoo}}{\emph{ticker}, \emph{date1}, \emph{date2}, \emph{cachename=None}, \emph{dividends=False}}{}
Fetch historical data for ticker between date1 and date2.  date1 and
date2 are date or datetime instances, or (year, month, day) sequences.

Ex:
fh = fetch\_historical\_yahoo(`\textasciicircum{}GSPC', (2000, 1, 1), (2001, 12, 31))

cachename is the name of the local file cache.  If None, will
default to the md5 hash or the url (which incorporates the ticker
and date range)

set dividends=True to return dividends instead of price data.  With
this option set, parse functions will not work

a file handle is returned

\end{fulllineitems}

\index{index\_bar() (in module newfinance)}

\begin{fulllineitems}
\phantomsection\label{newfinance:newfinance.index_bar}\pysiglinewithargsret{\code{newfinance.}\bfcode{index\_bar}}{\emph{ax}, \emph{vals}, \emph{facecolor='b'}, \emph{edgecolor='l'}, \emph{width=4}, \emph{alpha=1.0}}{}
Add a bar collection graph with height vals (-1 is missing).

ax          : an Axes instance to plot to
width       : the bar width in points
alpha       : bar transparency

\end{fulllineitems}

\index{parse\_yahoo\_historical() (in module newfinance)}

\begin{fulllineitems}
\phantomsection\label{newfinance:newfinance.parse_yahoo_historical}\pysiglinewithargsret{\code{newfinance.}\bfcode{parse\_yahoo\_historical}}{\emph{fh}, \emph{adjusted=True}, \emph{asobject=False}}{}
Parse the historical data in file handle fh from yahoo finance.
\begin{description}
\item[{\emph{adjusted}}] \leavevmode
If True (default) replace open, close, high, and low prices with
their adjusted values. The adjustment is by a scale factor, S =
adjusted\_close/close. Adjusted prices are actual prices
multiplied by S.

Volume is not adjusted as it is already backward split adjusted
by Yahoo. If you want to compute dollars traded, multiply volume
by the adjusted close, regardless of whether you choose adjusted
= True\textbar{}False.

\item[{\emph{asobject}}] \leavevmode
If False (default for compatibility with earlier versions)
return a list of tuples containing
\begin{quote}

d, open, close, high, low, volume
\end{quote}

If None (preferred alternative to False), return
a 2-D ndarray corresponding to the list of tuples.

Otherwise return a numpy recarray with
\begin{quote}

date, year, month, day, d, open, close, high, low,
volume, adjusted\_close
\end{quote}

where d is a floating poing representation of date,
as returned by date2num, and date is a python standard
library datetime.date instance.

The name of this kwarg is a historical artifact.  Formerly,
True returned a cbook Bunch
holding 1-D ndarrays.  The behavior of a numpy recarray is
very similar to the Bunch.

\end{description}

\end{fulllineitems}

\index{plot\_day\_summary() (in module newfinance)}

\begin{fulllineitems}
\phantomsection\label{newfinance:newfinance.plot_day_summary}\pysiglinewithargsret{\code{newfinance.}\bfcode{plot\_day\_summary}}{\emph{ax}, \emph{quotes}, \emph{ticksize=3}, \emph{colorup='k'}, \emph{colordown='r'}}{}
quotes is a sequence of (time, open, close, high, low, ...) sequences

Represent the time, open, close, high, low as a vertical line
ranging from low to high.  The left tick is the open and the right
tick is the close.

time must be in float date format - see date2num

ax          : an Axes instance to plot to
ticksize    : open/close tick marker in points
colorup     : the color of the lines where close \textgreater{}= open
colordown   : the color of the lines where close \textless{}  open
return value is a list of lines added

\end{fulllineitems}

\index{plot\_day\_summary2() (in module newfinance)}

\begin{fulllineitems}
\phantomsection\label{newfinance:newfinance.plot_day_summary2}\pysiglinewithargsret{\code{newfinance.}\bfcode{plot\_day\_summary2}}{\emph{ax}, \emph{opens}, \emph{closes}, \emph{highs}, \emph{lows}, \emph{ticksize=4}, \emph{colorup='k'}, \emph{colordown='r'}}{}
Represent the time, open, close, high, low as a vertical line
ranging from low to high.  The left tick is the open and the right
tick is the close.

ax          : an Axes instance to plot to
ticksize    : size of open and close ticks in points
colorup     : the color of the lines where close \textgreater{}= open
colordown   : the color of the lines where close \textless{}  open

return value is a list of lines added

\end{fulllineitems}

\index{quotes\_historical\_yahoo() (in module newfinance)}

\begin{fulllineitems}
\phantomsection\label{newfinance:newfinance.quotes_historical_yahoo}\pysiglinewithargsret{\code{newfinance.}\bfcode{quotes\_historical\_yahoo}}{\emph{ticker}, \emph{date1}, \emph{date2}, \emph{asobject=False}, \emph{adjusted=True}, \emph{cachename=None}}{}
Get historical data for ticker between date1 and date2.  date1 and
date2 are datetime instances or (year, month, day) sequences.

See {\hyperref[newfinance:newfinance.parse_yahoo_historical]{\code{parse\_yahoo\_historical()}}} for explanation of output formats
and the \emph{asobject} and \emph{adjusted} kwargs.

Ex:
sp = f.quotes\_historical\_yahoo(`\textasciicircum{}GSPC', d1, d2,
\begin{quote}

asobject=True, adjusted=True)
\end{quote}

returns = (sp.open{[}1:{]} - sp.open{[}:-1{]})/sp.open{[}1:{]}
{[}n,bins,patches{]} = hist(returns, 100)
mu = mean(returns)
sigma = std(returns)
x = normpdf(bins, mu, sigma)
plot(bins, x, color='red', lw=2)

cachename is the name of the local file cache.  If None, will
default to the md5 hash or the url (which incorporates the ticker
and date range)

\end{fulllineitems}

\index{volume\_overlay() (in module newfinance)}

\begin{fulllineitems}
\phantomsection\label{newfinance:newfinance.volume_overlay}\pysiglinewithargsret{\code{newfinance.}\bfcode{volume\_overlay}}{\emph{ax}, \emph{opens}, \emph{closes}, \emph{volumes}, \emph{colorup='k'}, \emph{colordown='r'}, \emph{width=4}, \emph{alpha=1.0}}{}
Add a volume overlay to the current axes.  The opens and closes
are used to determine the color of the bar.  -1 is missing.  If a
value is missing on one it must be missing on all

ax          : an Axes instance to plot to
width       : the bar width in points
colorup     : the color of the lines where close \textgreater{}= open
colordown   : the color of the lines where close \textless{}  open
alpha       : bar transparency

\end{fulllineitems}

\index{volume\_overlay2() (in module newfinance)}

\begin{fulllineitems}
\phantomsection\label{newfinance:newfinance.volume_overlay2}\pysiglinewithargsret{\code{newfinance.}\bfcode{volume\_overlay2}}{\emph{ax}, \emph{closes}, \emph{volumes}, \emph{colorup='k'}, \emph{colordown='r'}, \emph{width=4}, \emph{alpha=1.0}}{}
Add a volume overlay to the current axes.  The closes are used to
determine the color of the bar.  -1 is missing.  If a value is
missing on one it must be missing on all

ax          : an Axes instance to plot to
width       : the bar width in points
colorup     : the color of the lines where close \textgreater{}= open
colordown   : the color of the lines where close \textless{}  open
alpha       : bar transparency

nb: first point is not displayed - it is used only for choosing the
right color

\end{fulllineitems}

\index{volume\_overlay3() (in module newfinance)}

\begin{fulllineitems}
\phantomsection\label{newfinance:newfinance.volume_overlay3}\pysiglinewithargsret{\code{newfinance.}\bfcode{volume\_overlay3}}{\emph{ax}, \emph{quotes}, \emph{colorup='k'}, \emph{colordown='r'}, \emph{width=4}, \emph{alpha=1.0}}{}
Add a volume overlay to the current axes.  quotes is a list of (d,
open, close, high, low, volume) and close-open is used to
determine the color of the bar

kwarg
width       : the bar width in points
colorup     : the color of the lines where close1 \textgreater{}= close0
colordown   : the color of the lines where close1 \textless{}  close0
alpha       : bar transparency

\end{fulllineitems}



\chapter{stock module}
\label{stock::doc}\label{stock:module-stock}\label{stock:stock-module}\index{stock (module)}\index{stock (class in stock)}

\begin{fulllineitems}
\phantomsection\label{stock:stock.stock}\pysiglinewithargsret{\strong{class }\code{stock.}\bfcode{stock}}{\emph{cap}}{}
class for decribing stock evolution
\index{addHistoricaldata() (stock.stock method)}

\begin{fulllineitems}
\phantomsection\label{stock:stock.stock.addHistoricaldata}\pysiglinewithargsret{\bfcode{addHistoricaldata}}{\emph{currentVal=0}, \emph{time=0}}{}
\end{fulllineitems}

\index{bet() (stock.stock method)}

\begin{fulllineitems}
\phantomsection\label{stock:stock.stock.bet}\pysiglinewithargsret{\bfcode{bet}}{\emph{betVal}}{}
\end{fulllineitems}

\index{evolve() (stock.stock method)}

\begin{fulllineitems}
\phantomsection\label{stock:stock.stock.evolve}\pysiglinewithargsret{\bfcode{evolve}}{\emph{change=0}, \emph{use=False}}{}
\end{fulllineitems}

\index{getAll() (stock.stock method)}

\begin{fulllineitems}
\phantomsection\label{stock:stock.stock.getAll}\pysiglinewithargsret{\bfcode{getAll}}{}{}
\end{fulllineitems}

\index{getCap() (stock.stock method)}

\begin{fulllineitems}
\phantomsection\label{stock:stock.stock.getCap}\pysiglinewithargsret{\bfcode{getCap}}{}{}
\end{fulllineitems}

\index{getTime() (stock.stock method)}

\begin{fulllineitems}
\phantomsection\label{stock:stock.stock.getTime}\pysiglinewithargsret{\bfcode{getTime}}{}{}
\end{fulllineitems}

\index{getTimes() (stock.stock method)}

\begin{fulllineitems}
\phantomsection\label{stock:stock.stock.getTimes}\pysiglinewithargsret{\bfcode{getTimes}}{}{}
\end{fulllineitems}

\index{m\_cap (stock.stock attribute)}

\begin{fulllineitems}
\phantomsection\label{stock:stock.stock.m_cap}\pysigline{\bfcode{m\_cap}\strong{ = 0}}
\end{fulllineitems}

\index{m\_iters (stock.stock attribute)}

\begin{fulllineitems}
\phantomsection\label{stock:stock.stock.m_iters}\pysigline{\bfcode{m\_iters}\strong{ = 0}}
\end{fulllineitems}

\index{m\_time (stock.stock attribute)}

\begin{fulllineitems}
\phantomsection\label{stock:stock.stock.m_time}\pysigline{\bfcode{m\_time}\strong{ = 0}}
\end{fulllineitems}

\index{m\_time\_his (stock.stock attribute)}

\begin{fulllineitems}
\phantomsection\label{stock:stock.stock.m_time_his}\pysigline{\bfcode{m\_time\_his}\strong{ = {[}{]}}}
\end{fulllineitems}

\index{m\_val (stock.stock attribute)}

\begin{fulllineitems}
\phantomsection\label{stock:stock.stock.m_val}\pysigline{\bfcode{m\_val}\strong{ = {[}{]}}}
\end{fulllineitems}

\index{next() (stock.stock method)}

\begin{fulllineitems}
\phantomsection\label{stock:stock.stock.next}\pysiglinewithargsret{\bfcode{next}}{}{}
\end{fulllineitems}


\end{fulllineitems}



\chapter{test module}
\label{test:module-test}\label{test::doc}\label{test:test-module}\index{test (module)}

\chapter{testVirtualMarket module}
\label{testVirtualMarket:testvirtualmarket-module}\label{testVirtualMarket::doc}

\chapter{virtualMarket module}
\label{virtualMarket:virtualmarket-module}\label{virtualMarket::doc}

\chapter{Indices and tables}
\label{index:indices-and-tables}\begin{itemize}
\item {} 
\emph{genindex}

\item {} 
\emph{modindex}

\item {} 
\emph{search}

\end{itemize}


\renewcommand{\indexname}{Python Module Index}
\begin{theindex}
\def\bigletter#1{{\Large\sffamily#1}\nopagebreak\vspace{1mm}}
\bigletter{c}
\item {\texttt{conf}}, \pageref{conf:module-conf}
\indexspace
\bigletter{m}
\item {\texttt{market}}, \pageref{market:module-market}
\indexspace
\bigletter{n}
\item {\texttt{newfinance}}, \pageref{newfinance:module-newfinance}
\indexspace
\bigletter{s}
\item {\texttt{stock}}, \pageref{stock:module-stock}
\indexspace
\bigletter{t}
\item {\texttt{test}}, \pageref{test:module-test}
\end{theindex}

\renewcommand{\indexname}{Index}
\printindex
\end{document}
